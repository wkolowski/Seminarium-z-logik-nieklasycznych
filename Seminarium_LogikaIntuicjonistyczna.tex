\documentclass{beamer}
\usepackage[utf8]{inputenc}
\usepackage{polski}
\usepackage[polish]{babel}
\usetheme{Darmstadt}

\newcommand{\impl}{\rightarrow}
\renewcommand{\implies}{\rightarrow}

\title{Logika intuicjonistyczna}
 
\author{Zeimer}
\date{6, 13 marca 2018}

\begin{document}

\frame{\titlepage}

\frame{\tableofcontents}

\section{Filozofia}

\begin{frame}{Kto i po co wymyśla logiki}
\begin{itemize}
	\item Filozofowie — ''A może poszli do lasu?'' — potrzebna jest logika modalna, która rozstrzygnie, czy poszli, czy może jest konieczne, że jednak nie.
	\item Lingwiści — ''Polacy mordowali żydów.'' — potrzebna jest jakaś logika, w której będzie można formalizować znaczenie zdań języka naturalnego.
	\item Informatycy — ''U mnie działa.'' — potrzebna jest jakaś logika, która pozwoli chwycić programy za mordę i sprawi, że będą działały u wszystkich.
	\item Matematycy — wymyślają jakieś głupoty z nudów, żeby brać na nie granty
\end{itemize}
\end{frame}

\begin{frame}{Jak powstaje konkretna logika — teoria}
\begin{itemize}
	\item Problem — ktoś ma jakiś problem i uważa, że można go rozwiązać za pomocą logiki.
	\item Intuicje — twórca logiki ma pewien zestaw intuicji, pozwalający mu nieformalnie rozumować na temat problemu.
	\item Składnia — intuicyjne myślenie o problemie wyraża się w pewnym języku nieformalnym. Żeby wyrazić formalne myślenie o problemie, trzeba skonstruować do tego celu język formalny.
	\item Semantyka — próba uchwycenia ulotnych i niewyraźnych intuicji za pomocą precyzyjnego języka matematyki.
	\item System(y) dowodzenia — próba zastąpienia dowodów intuicyjnych (niepewne) i opartych na semantyce (trudne w automatyzacji) przez formalne manipulacje na symbolach.
\end{itemize}
\end{frame}

\begin{frame}{Jak powstaje logika — praktyka}
W praktyce nie jest tak kolorowo jak na powyższym slajdzie.
\begin{itemize}
	\item Logiki często nie powstają, by rozwiązać jakiś problem, ale dlatego, że ich autor nie lubi jakiejś innej logiki i postanowił zrobić swoją (np. Brouwer, prekursor logiki intuicjonistycznej).
	\item Intuicji często brak, jeżeli logika wiążę się głównie z dziwnymi zabawami matematyków (np. wyrzućmy coś z jakiejś logiki i zobaczmy, jakie twierdzenia da się udowodnić).
	\item Składnia zazwyczaj zostaje ukradziona z logiki klasycznej lub jest jej lekką modyfikacją.
	\item Dobrej semantyki często brak. Jest tak w przypadku logiki intuicjonistycznej (choć nie do końca) oraz logiki liniowej (tutaj sytuacja jest beznadziejna).
\end{itemize}
\end{frame}

\begin{frame}{Cele rozwoju logiki}
Rozwijanie konkretnej logiki ma różne cele:
\begin{itemize}
	\item Formalny: celem rozwoju logiki rozumianej tutaj jako pary (semantyka, system dowodzenia) jest udowodnienie:
	\begin{itemize}
		\item Twierdzenia o poprawności (ang. soundness), które mówi, że możemy dowieść tylko zdań prawdziwych.
		\item Twierdzenia o pełności (ang. completeness), które mówi, że każde zdanie prawdziwe ma dowód.
	\end{itemize}
		Pierwsze z nich jest absolutnie konieczne. Drugie jest miłe, ale czasem nieosiągalne.
	\item Filozoficzny: wymyślanie nowych semantyk w celu pogłębienia wcześniejszych intuicji.
	\item Inżynierski: wymyślanie nowych systemów dowodzenia, aby łatwiej było dowodzić.
	\item Praktyczny: rozwiązywanie problemów, do których stworzono logikę.
\end{itemize}
\end{frame}

\section{Historia}

\begin{frame}{Jak powstała logika intuicjonistyczna}
\begin{itemize}
	\item Zaczęło się od tego, że niejaki L. E. J. Brouwer z różnych filozoficznych powodów nie lubił logiki klasycznej. W swoich pracach potępiał on prawo wyłączonego środka. Tak powstała intuicja.
	\item Jego uczeń Arend Heyting ukradł składnię z logiki klasycznej i wymyślił algebry Heytinga. Tak powstała składnia i pierwsza semantyka.
	\item Gerhard Gentzen wymyślił dedukcję naturalną i rachunek sekwentów. Tak powstały systemy dowodzenia.
	\item Kripke wymyślił semantykę Kripkego. Tak powstała kolejna semantyka.
	\item Curry i Howard zauważyli, że dowody w logice intuicjonistycznej odpowiadają termom rachunku lambda z typami prostymi i że można wykorzystać to do komputerowej formalizacji dowodów. Tak powstał problem, który logika intuicjonistyczna rozwiązywała.
\end{itemize}
\end{frame}

\section{Intuicje}

\begin{frame}{Intuicje — potęgowanie liczb niewymiernych}
\begin{theorem}
Istnieją takie liczby niewymierne $a, b$, że $a^b$ jest wymierne.
\end{theorem}
\begin{proof}
Rozważmy przypadki: \\
1. Jeżeli $\sqrt{2}^{\sqrt{2}}$ jest wymierne, niech $a = b = \sqrt{2}$. Wtedy $a^b = \sqrt{2}^{\sqrt{2}}$ jest wymierne na mocy założenia. \\
2. Jeżeli nie, to niech $a = \sqrt{2}^{\sqrt{2}}$, b = $\sqrt{2}$. Wtedy $a^b = (\sqrt{2}^{\sqrt{2}})^{\sqrt{2}} = \sqrt{2}^{\sqrt{2} \cdot \sqrt{2}} = \sqrt{2}^2 = 2$ jest wymierne.
\end{proof}
\end{frame}

\begin{frame}{Intuicje — potęgowanie liczb niewymiernych}
Zagadka: podaj takie liczby niewymierne $a, b$, że $a^b$ jest wymierne.
Rozwiązanie zagadki: mimo że na poprzednim slajdzie udowodniliśmy, że takie liczby istnieją, to nie wiemy, co to za liczby. Wynika to z faktu, że posłużyliśmy się zasadą wyłączonego środka (ang. law of excluded middle, w skrócie LEM), głoszącą, że $P \lor \neg P$. Jest ona niekonstruktywna, tzn. pozwala udowodnić istnienie obiektów bez konstruowania ich wprost.
\end{frame}

\begin{frame}{Intuicje — paradoks pijoka}
\begin{theorem}
W każdym niepustym barze istnieje taka osoba (nazwijmy ją pijokiem), że jeżeli ona pije, to wszyscy piją.
\end{theorem}
\begin{proof}
Rozważmy przypadki: \\
1. Jeżeli wszyscy piją, to pozamiatane (bar jest niepusty, więc na pijoka weźmy kogokolwiek). \\
2. Jeżeli ktoś nie pije, to niech on zostanie pijokiem. Załóżmy, że pijok pije. Ponieważ jednak wiemy, że pijok nie pije, to dostajemy sprzeczność i wobec tego wszyscy piją.
\end{proof}
\end{frame}

\begin{frame}{Intuicje — paradoks pijoka}
Zagadka: wejdź do swojego ulubionego baru i wskaż pijoka.
Rozwiązanie zagadki: mamy tutaj ten sam problem, co poprzednio. Mimo iż udowodniliśmy istnienie pijoka, to nie wiemy, kto konkretnie nim jest. Zasada wyłączonego środka i niekonstruktywizm po raz kolejny weszły nam w paradę.
\end{frame}

\begin{frame}{Intuicje — prawo wyłączonego środka}
Problemy z LEM staną się jaśniejsze, gdy naszą logikę będziemy rozważać w bardziej obliczeniowy sposób, mianowicie gdy dowód $P$ będziemy rozumieć jako program, który jest ''certyfikatem'' prawdziwości $P$. W takim układzie dowód $P \lor \neg P$ to program rozstrzygający prawdziwość zdania $P$. W takiej interpretacji LEM pozwala nam rozstrzygnąć wszystkie problemy nierozstrzygalne.
\end{frame}

\begin{frame}{Intuicje — prawo wyłączonego środka}
\begin{itemize}
	\item 1. Problem stopu. Na mocy LEM każda maszyna Turinga kończy pracę albo i nie, ale nie potrafimy sprawdzić, która z tych dwóch opcji zachodzi.
	\item 2. Równość liczb rzeczywistych. LEM mówi nam, że dwie liczby rzeczywisty albo są równe, albo nie. Tego również nie potrafimy sprawdzić (jeżeli faktycznie są równe i mają nieskończone rozwinięcia dziesiętne, to porównywanie ich cyfra po cyfrze nigdy się nie skończy).
\end{itemize}
\end{frame}

\section{Składnia}

\begin{frame}{Składnia}

{\centering
$\phi, \psi ::= \top | \bot | p | \neg \phi | \phi \lor \psi | \phi \land \psi | \phi \implies \psi | \phi \iff \psi$ \\
}
\relax

\begin{itemize}
	\item Składnia logiki intuicjonistycznej jest taka sama jak składnia logiki klasycznej ($p$ oznacza tutaj jedną z przeliczalnie wielu zmiennych zdaniowych).
	
	\item Żeby zmniejszyć rozmiar składni, czynimy następujące uproszczenia:
	\begin{itemize}
		\item Wyrzucamy $\neg \phi$ i definiujemy $\neg \phi = \phi \implies \bot$
		\item Wyrzucamy $\phi \iff \psi$ i definiujemy $\phi \iff \psi = \phi \implies \psi \land \psi \implies \phi$
	\end{itemize}
	\item Są jeszcze inne możliwości uproszczenia składni, których jednak nie poczynimy:
	\begin{itemize}
		\item Moglibyśmy wyrzucić $\top$ i zdefiniować $\top = \bot \implies \bot$
	\end{itemize}
\end{itemize}

\end{frame}

\section{Semantyki}

\begin{frame}{Semantyka formalna, nieformalna i intuicje}
Oprócz intuicji oraz semantyki formalnej jest też trzeci byt, który możemy nazwać semantyką nieformalną. Intuicja jest nieformalna i przedskładniowa, zaś semantyka formalna jest formalna i poskładniowa. Semantyka nieformalna jest nieformalna i poskładniowa. \\~\

W przypadku logiki intuicjonistycznej pochodzi ona od rodziny systemów dowodzenia zwanych zbiorczo dedukcją naturalną. Dzieje się tak w myśl maksymy Wittgensteina ''meaning is use'', czyli ''znaczenie to użycie'' - nieformalna semantyka logiki intuicjonistycznej pochodzi od reguł wnioskowania rządzących dedukcją naturalną, z którą zapoznamy się później. \\~\
\end{frame}

\begin{frame}{Algebry Heytinga - definicja}
$(H, \leq, \top, \bot, \land, \lor, \impl)$ jest algebrą Heytinga, gdy $(H, \leq)$ jest częściowym porządkiem z elementem najmniejszym $\bot$ i największym $\top$, zaś $\land$, $\lor$ i $\impl$ to działania binarne i dla dowolnych $x, y, z \in H$ zachodzi: \\
\begin{itemize}
	\item $x \leq \top$
	\item $x \land y \leq x$
	\item $x \land y \leq y$
	\item $z \leq x$ i $z \leq y$ implikuje $z \leq x \land y$
	\item $\bot \leq x$
	\item $x \leq x \lor y$
	\item $y \leq x \lor y$
	\item $x \leq z$ i $y \leq z$ implikuje $x \lor y \leq z$
	\item $z \leq (x \impl y)$ wtedy i tylko wtedy, gdy $z \land x \leq y$
\end{itemize}
\end{frame}

\begin{frame}{Algebry Heytinga - waluacje}
Niech $H$ będzie algebrą Heytinga. \\~\

Waluacja to funkcja $v : \text{Var} \to H$, gdzie $\text{Var}$ oznacza zbiór zmiennych zdaniowych naszej logiki. \\~\

Waluację możemy rozszerzyć do funkcji $\tilde{v}: \text{Frm} \to H$, gdzie $\text{Frm}$ to zbiór zdań naszej logiki, za pomocą następującej definicji: \\
\begin{center}
$\tilde{v}(\top) = \top$ \\
$\tilde{v}(\bot) = \bot$ \\
$\tilde{v}(P \land Q) = \tilde{v}(P) \land \tilde{v}(Q)$ \\
$\tilde{v}(P \lor Q) = \tilde{v}(P) \lor \tilde{v}(Q)$ \\
$\tilde{v}(P \impl Q) = \tilde{v}(P) \impl \tilde{v}(Q)$ \\
\end{center}

Zauważmy, że symbole $\top, \bot, \land, \lor, \impl$ po lewej oznaczają spójniki logiczne, a po prawej - operacje w algebrze Heytinga $H$.
\end{frame}

\begin{frame}{Algebry Heytinga - spełnianie}
Niech $\Gamma = \{\phi_1, \dots, \phi_n\}$ będzie skończonym zbiorem zdań. \\
Niech $\bigwedge \Gamma$ oznacza $\phi_1 \land \dots \land \phi_n$. \\
Niech $\psi$ będzie zdaniem naszej logiki. \\~\

$\psi$ jest $H$-konsekwencją zbioru zdań $\Gamma$, co zapisujemy $\Gamma \models_H \psi$, gdy dla każdej waluacji $v: \text{Var} \to H$ zachodzi $\tilde{v}(\bigwedge \Gamma) \leq \tilde{v}(\psi)$. \\~\

Jako specjalny przypadek powyższej definicji możemy powiedzieć, że $\psi$ jest spełnione w $H$ (ang. $H$-valid), co zapisujemy $\models_H \psi$, gdy dla każdej waluacji $v: \text{Var} \to H$ mamy $\tilde{v}(\psi) = \top$.
\end{frame}

\begin{frame}{Algebry Heytinga - fakty}
\begin{theorem}
Semantyka algebro-Heytingowa jest poprawna i pełna (ang. sound and complete), tzn. zdanie $\phi$ jest dowodliwe w systemie dedukcji naturalnej z kontekstami wtedy i tylko wtedy, gdy jest spełnione w każdej algebrze Heytinga $H$.
\end{theorem}
\begin{theorem}
Zdanie $\phi$ jest dowodliwe w systemie dedukcji naturalnej z kontekstami wtedy i tylko wtedy, gdy jest spełnione w każdej \textbf{skończonej} algebrze Heytinga $H$.
\end{theorem}
\end{frame}

\begin{frame}{Spostrzeżenie filozoficzne}
Zauważmy, że powyższe twierdzenia głoszą, że semantyka jest poprawna i pełna względem jakiegoś systemu dowodzenia, nie zaś na odwrót, jak teoretycznie powinno być. Jest to znamienne, gdyż w logice intuicjonistycznej dominuje wspomniana już semantyka nieformalna, pochodząca z systemu dedukcji naturalnej.
\end{frame}

\begin{frame}{Semantyka Kripkego}
Ramka Kripkego to para $(W, R)$, gdzie $W$ jest zbiorem, zaś $R$ relacją binarną na $W$. Elementy $W$ będziemy nazywać światami, zaś $R$ będziemy nazywać relacją dostępności. \\~\

Modelem Kripkego będziemy nazywać trójkę $(W, \leq, \Vdash)$, gdzie $(W, \leq)$ jest ramką Kripkego, w której $\leq$ jest preporządkiem, zaś $\Vdash$ jest relacją między światami a zdaniami naszej logiki, która dla dowolnej zmiennej zdaniowej $p$ oraz zdań $\phi$ i $\psi$ spełnia następujące warunki: \\
\begin{center}
$w \Vdash \top$ \\
$w \leq u$ i $w \Vdash p$ implikuje $u \Vdash p$ \\
$w \Vdash \phi \land \psi$ wtw $w \Vdash \phi$ i $w \Vdash \psi$ \\
$w \Vdash \phi \lor \psi$ wtw $w \Vdash \phi$ lub $w \Vdash \psi$ \\
$w \Vdash \phi \impl \psi$ wtw dla każdego $u \geq w$ jeżeli $u \Vdash \phi$ to $u \Vdash \psi$ \\
$w \Vdash \bot$ nie zachodzi
\end{center}
\end{frame}

\begin{frame}{Semantyka Kripkego - intuicje}
Semantyka Kripkego tłumaczy prawdziwość zdań w logice intuicjonistycznej w sposób temporalno-epistemiczny, tj. mówi jak matematycy zdobywają więdzę w czasie: \\
\begin{itemize}
	\item Światy możemy utożsamiać ze stanami wiedzy. \\
	\item Zapis $w \Vdash \phi$ możemy odczytywać tak, że $\phi$ jest prawdziwe w świecie $w$. \\
	\item Zapis $w \leq u$ możemy odczytywać tak, że $u$ jest światem większej wiedzy niż $w$.
\end{itemize}
\end{frame}

\begin{frame}{Semantyka Kripkego - fakty}
\begin{theorem}
Semantyka Kripkego jest poprawna i pełna (ang. sound and complete), tzn. zdanie $\phi$ jest dowodliwe w systemie dedukcji naturalnej z kontekstami wtedy i tylko wtedy, gdy $\phi$ zachodzi w każdym świecie każdego modelu Kripkego.
\end{theorem}
\end{frame}

\section{Systemy dowodzenia}

\begin{frame}{Język i metajęzyk}
Należy rozróżnić pojęcia języka i metajęzyka. Przez język należy rozumieć logikę, którą się zajmujemy (w naszym przypadku jest to logika intuicjonistyczna), zaś metajęzyk to język, w którym przedstawiamy naszą logikę (w naszym wypadku jest to język polski). \\~\

Rozróżnienie to jest ważne, gdyż istnienie zdań mówiących coś o samych sobie może prowadzić do sprzeczności, a tego nie chcemy. Żeby się przed tym bronić, o zdaniach języka będziemy zawsze mówić w metajęzyku. \\~\

To rozróżnienie pomoże nam też zrozumieć, jak działają systemy dowodzenia.
\end{frame}

\begin{frame}{Osąd}
	Osąd (ang. judgement) - zdanie w metajęzyku mówiące coś o zdaniach języka. Przykładem osądu może być stwierdzenie ''zdanie $P \land Q$ jest prawdziwe'' - jest to polskie zdanie mówiące coś o zdaniu $P \land Q$, będące zdaniem naszej logiki intuicjonistycznej. \\~\
	
	Osąd ''zdanie jest prawdziwe'' jest chyba najczęściej badanym, ale mogą być też inne: ''zdanie jest fałszywe'', ''zdanie jest poprawnie zbudowane'' etc. Bardzo ważnym osądem jest osąd hipotetyczny (ang. hypothetical judgement), który ma postać ''ze zdania $P$ wynika zdanie $Q$''. \\~\
	
	Zauważmy, że osądy mogą być prawdziwe i fałszywe. Przykładem fałszywego osądu może być osąd ''zdanie $\bot$ jest prawdziwe'' (chyba, że logika którą formalizujemy jest sprzeczna).
\end{frame}

\begin{frame}{Reguła wnioskowania}
	Reguła wnioskowania (ang. rule of inference) - w najogólniejszym rozumieniu jest to pewna operacja zdefiniowana w metajęzyku, za pomocą której można przekształcać kolekcje osądów w inne kolekcje osądów. \\~\
	
	Niech osąd $P \ \text{true}$ oznacza ''zdanie $P$ jest prawdziwe''. Przykładem reguły wnioskowania może być reguła wprowadzania koniunkcji, którą możemy zapisać \\

	\begin{center}
		$\displaystyle \frac{P \ \text{true} \qquad Q \ \text{true}}{P \land Q \ \text{true}}$
	\end{center}
	
	Reguła, mimo że zapisana za pomocą symboli, jest zdaniem języka polskiego, które głosi, że jeżeli zdanie $P$ jest prawdziwe i zdanie $Q$ jest prawdziwe, to zdanie $P \land Q$ jest prawdziwe.
\end{frame}

\begin{frame}{Reguły wnioskowania - konwencje}
	Wygodną konwencją dotyczącą zapisu reguł wnioskowania jest to, że jeżeli w danym systemie dowodzenia występuje tylko jeden rodzaj osądów, to zamiast osądu będziemy pisać zdanie, którego on dotyczy, np. zamiast osądu $P \ \text{true}$ będziemy pisać samo $P$. W tej konwencji regułę z poprzedniego slajdu możemy przedstawić tak: \\
	
	\begin{center}
		$\displaystyle \frac{P \quad Q}{P \land Q}$
	\end{center}
	
	Konwencję tę można stosować też, gdy jest więcej niż jeden rodzaj osądów, ale nie istnieje ryzyko pomyłki.
\end{frame}

\begin{frame}{Aksjomat}
	Aksjomat to osąd, który uznajemy za prawdziwy bez dowodu. Przykładem aksjomatu może być np. osąd $P \lor \neg P \ \text{true}$, który głosi, że ''zdanie $P \lor \neg P$ jest prawdziwe''.
\end{frame}

\begin{frame}{Aksjomaty a reguły}
	Aksjomat z poprzedniego slajdu może nieco przypominać regułę wnioskowania \\
	
	\begin{center}
		$\displaystyle \frac{}{P \lor \neg P \ \text{true}}$
	\end{center}
	
	która głosi, że z nicości możemy wywnioskować prawdziwość zdania $P \lor \neg P$. Różnica między aksjomatami i regułami staje się istotna, gdy próbujemy nadać naszej logice interpretację obliczeniową. Te konstrukty, które mają taką interpretację, zostają wtedy regułami wnioskowania, aksjomaty służą zaś do wyrażania osądów niemających interpretacji obliczeniowej.
\end{frame}

\begin{frame}{System dowodzenia}
	System dowodzenia (ang. proof system) dla logiki (rozumianej tutaj jako czysta składnia, lub też po prostu zbiór zdań $\text{Frm}$) to kolekcja osądów, reguł wnioskowania i aksjomatów. Regułom i aksjomatom nadajemy zazwyczaj jakieś nazwy, żeby można było łatwo się do nich odnosić. System może też wprowadzać jakieś nowe rodzaje bytów (np. konteksty), jeżeli są potrzebne do wydawania osądów. \\~\
	
	O ile wszystkie systemy dowodzenia opierają się na osądach, o tyle reguły wnioskowania i aksjomaty mogą występować w różnych proporcjach.
\end{frame}

\begin{frame}{System Hilberta dla logiki intuicjonistycznej}
	Osąd: $P \ \text{true}$ - ''zdanie $P$ jest prawdziwe'' (zapisywany wg naszej konwencji po prostu jako $P$).
	
	Reguła wnioskowania: $\displaystyle \frac{\phi \implies \psi \qquad \phi}{\psi}$ (modus ponens)
	
	Aksjomaty:

	\begin{itemize}
		\item THEN-1: $\phi \to (\chi \to \phi)$
		\item THEN-2: $(\phi \to (\chi \to \psi)) \to ((\phi \to \chi )\to (\phi \to \psi ))$
		\item AND-1: $\phi \land \chi \to \phi$
		\item AND-2: $\phi \land \chi \to \chi$
		\item AND-3: $\phi \to (\chi \to (\phi \land \chi))$
		\item OR-1: $\phi \to \phi \lor \chi$
		\item OR-2: $\chi \to \phi \lor \chi$
		\item OR-3: $(\phi \to \psi )\to ((\chi \to \psi )\to (\phi \lor \chi \to \psi ))$
		\item FALSE: $\bot \to \phi$
		\item TRUE: $\top$
	\end{itemize}
\end{frame}

\begin{frame}{Systemy Hilberta}
	Przedstawiony powyżej system Hilberta jest dość minimalistyczny - ma jeden osąd, jedną regułę wnioskowania i 10 aksjomatów. Jest nieporęczny zarówno dla ludzi (trudno dowodzić w nim ręcznie, gdyż różni się od nieformalnego sposobu dowodzenia w matematyce) jak i dla komputerów (nieprzydatny w automatycznym dowodzeniu). \\~\
	
	Zauważmy, że nie jest to jedyny system Hilberta, jaki można skonstruować - jest ich nieskończenie wiele, w zależności od wybranego zestawu aksjomatów. 
\end{frame}

\begin{frame}{Rachunek sekwentów (dla logiki klasycznej)}
	Rachunek sekwentów to system, w którym występuje jeden osąd, zwany sekwentem. Zapisujemy go $\Gamma \vdash \Delta$, a czytamy ''z koniunkcji zdań ze zbioru $\Gamma$ wynika alternatywa zdań ze zbioru $\Delta$''. \\~\
	
	Reguły wnioskowania występują w dwóch seriach, nazywanych lewą i prawą. ''Lewe'' reguły mówią, jak operować na zdaniach, które w sekwencie są po lewej stronie, a ''prawe'' jak operować na zdaniach, które w sekwencie są po prawej. \\~\
	
	Charakterystyczne dla tego systemu jest, że reguł wnioskowania jest dużo, nie ma zaś aksjomatów.
\end{frame}

\begin{frame}{Klasyczny rachunek sekwentów - spójniki}
\begin{columns}
\begin{column}{0.6\textwidth}
\begin{itemize}
	\setlength\itemsep{2em}
	\item $\displaystyle \frac{\Gamma, A \land B \vdash \Delta}{\Gamma, A, B \vdash \Delta} (L\land)$
	\item $\displaystyle \frac{\Gamma, A \lor B \vdash \Delta}{\Gamma, A \vdash \Delta \qquad \Gamma, B \vdash \Delta} (L\lor)$
	\item $\displaystyle \frac{\Gamma, \lnot A \vdash \Delta}{\Gamma \vdash \Delta, A} (L\lnot)$
	\item $\displaystyle \frac{\Gamma, A \rightarrow B \vdash \Delta}{\Gamma \vdash \Delta, A \qquad \Gamma, B \vdash \Delta} (L\rightarrow)$
\end{itemize}
\end{column}
\begin{column}{0.5\textwidth}
\begin{itemize}
	\setlength\itemsep{2em}
	\item $\displaystyle \frac{\Gamma \vdash \Delta, A \land B}{\Gamma \vdash \Delta, A \qquad \Gamma \vdash \Delta, B} (R\land)$
	\item $\displaystyle \frac{\Gamma \vdash \Delta, A \lor B}{\Gamma \vdash \Delta, A, B} (R\lor)$
	\item $\displaystyle \frac{\Gamma \vdash \Delta, \lnot A}{\Gamma, A \vdash \Delta} (R\lnot)$
	\item $\displaystyle \frac{\Gamma \vdash \Delta, A \rightarrow B}{\Gamma, A \vdash \Delta, B} (R\rightarrow)$
\end{itemize}
\end{column}
\end{columns}
\end{frame}

\begin{frame}{Interpretacja reguł dla spójników}
	Regułę postaci \\
	
	\begin{center}
		$\displaystyle \frac{\mathcal{J}_1 \dots \mathcal{J}_n}{\mathcal{J}'_1 \dots \mathcal{J}'_n}$
	\end{center}
	
	można interpretować (czytając z góry na dół): ''żeby dowieść osądów $\mathcal{J}_1, \dots, \mathcal{J}_n$, wystarczy dowieść osądów $\mathcal{J}'_1, \dots, \mathcal{J}'_n$. \\~\
	
	Zauważmy, że reguły dla negacji możemy otrzymać z reguł dla implikacji, zaś reguły dla implikacji z reguł dla negacji i dysjunkcji.
\end{frame}

\begin{frame}{Klasyczny rachunek sekwentów - reguły strukturalne}
\begin{columns}
\begin{column}{0.6\textwidth}
\begin{itemize}
	\setlength\itemsep{2em}
	\item $\displaystyle \frac{\Gamma, A \vdash \Delta}{\Gamma \vdash \Delta}$ (WL)
	\item $\displaystyle \frac{\Gamma, A \vdash \Delta}{\Gamma, A, A \vdash \Delta}$ (CL)
	\item $\displaystyle \frac{\Gamma_1, A, B, \Gamma_2 \vdash \Delta}{\Gamma_1, B, A, \Gamma_2 \vdash \Delta}$ (PL)
\end{itemize}
\end{column}
\begin{column}{0.5\textwidth}
\begin{itemize}
	\setlength\itemsep{2em}
	\item $\displaystyle \frac{\Gamma \vdash A, \Delta}{\Gamma \vdash \Delta}$ (WR)
	\item $\displaystyle \frac{\Gamma \vdash A, \Delta}{\Gamma \vdash A, A,\Delta}$ (CR)
	\item $\displaystyle \frac{\Gamma \vdash \Delta_1, A, B, \Delta_2}{\Gamma \vdash \Delta_1, B, A, \Delta_1}$ (PR)
\end{itemize}
\end{column}
\end{columns}
\end{frame}

\begin{frame}{Interpretacja reguł strukturalnych}
	Reguły WL i WR mówią, że możemy usuwać założenia i cele; CL i CR mówią, że możemy je kopiować; PL i PR mówią, że możemy zamieniać ich kolejność. \\~\
	
	Nie są to jednak reguły, które można stosować automatycznie, gdyż moglibyśmy nieopatrznie usunąć sobie potrzebne założenie, albo wpaść w nieskończoną pętlę, na ślepo kopiując i zamieniając kolejność założeń i celów.
\end{frame}

\begin{frame}{Klasyczny rachunek sekwentów - pozostałe reguły}
	\begin{center}
		$\displaystyle \frac{}{A \vdash A} (\text{Ass})$
	\end{center}
	
	Ass to reguła, która pozwala nam skorzystać z założenia.
	
	\begin{center}
		$\displaystyle \frac{\Gamma, \Sigma \vdash \Delta, \Pi}{\Gamma \vdash \Delta, A \qquad A, \Sigma \vdash \Pi}$ (Cut)
	\end{center}
	
	Cut to reguła, która pozwala nam udowodnić na boku potrzebny nam lemat. \\~\
	
	Uwaga: w powyższej prezentacji pominięto reguły dla $\top$ oraz $\bot$, gdyż są mało ciekawe.
\end{frame}

\begin{frame}{Klasyczny rachunek sekwentów - podsumowanie}
	Zauważmy, że stosowanie reguł dla spójników daje w wyniku sekwenty o mniejszej złożoności. Dzięki temu możemy je stosować bez zastanowienia, co sprawia, że rachunek sekwentów świetnie sprawdza się w automatycznym dowodzeniu. \\~\
	
	Rachunek sekwentów dla logiki intuicjonistycznej można uzyskać z tego zaprezentowanego powyżej poprzez ograniczenie zbioru formuł po prawej stronie sekwentu do singletonu. \\~\
	
	\url{http://logitext.mit.edu/main}
	 to świetne miejsce, gdzie można poćwiczyć dowodzenie w rachunku sekwentów (zarówno klasycznym, jak i intuicjonistycznym). Uwaga: zabawa jest dość bezmyślna.
\end{frame}

\begin{frame}{Dedukcja naturalna}
	Dedukcja naturalna to system, który powstał, by oddać sposób, w jaki matematycy rozumują na co dzień. \\~\
	
	Jedynym osądem jest osąd hipotetyczny, zapisywany jako $\Gamma \vdash P$, który możemy interpretować ''z koniunkcji zdań ze zbioru $\Gamma$ wynika zdanie $P$''. \\~\
	
	Podobnie jak w rachunku sekwentów mamy tu sporo reguł i żadnych aksjomatów. \\~\
	
	Systemy dedkucji naturalnej występują w wielu wersjach: Fitch, tradycyjnym (bez kontekstów) oraz z kontekstami. Przyjrzymy się tylko ostatniemu z nich, gdyż jest on najbardziej praktyczny.
\end{frame}

\begin{frame}{Dedukcja naturalna - reguły dla spójników}
\begin{columns}
	\begin{column}{0.6\textwidth}
	\begin{itemize}
		\setlength\itemsep{2em}
		\item $\displaystyle \frac{\Gamma \vdash A \qquad \Gamma \vdash B}{\Gamma \vdash A \land B} (\land\text{-intro})$
		\item $\displaystyle \frac{\Gamma \vdash A}{\Gamma \vdash A \lor B} (\lor\text{-intro L})$
		\item $\displaystyle \frac{\Gamma \vdash B}{\Gamma \vdash A \lor B} (\lor\text{-intro R})$
		\item $\displaystyle \frac{\Gamma, A \vdash B}{\Gamma \vdash A \implies B} (\implies\text{-intro})$
	\end{itemize}
	\end{column}
	
	\begin{column}{0.6\textwidth}
	\begin{itemize}
		\setlength\itemsep{2em}
		\item $\displaystyle \frac{\Gamma \vdash A \land B}{\Gamma \vdash A} (\land\text{-elim L})$
		\item $\displaystyle \frac{\Gamma \vdash A \land B}{\Gamma \vdash B} (\land\text{-elim R})$
		\item $\displaystyle \frac{\Gamma, A \vdash C \qquad \Gamma, B \vdash C}{\Gamma, A \lor B \vdash C} (\lor\text{-elim})$
		\item $\displaystyle \frac{\Gamma \vdash A \impl B \qquad \Gamma \vdash A}{\Gamma \vdash B} (\impl\text{-elim})$
	\end{itemize}
	\end{column}
\end{columns}
\end{frame}

\begin{frame}{Interpretacja reguł dla spójników}
	Reguły wnioskowania dla spójników dzielą się na dwie grupy, mianowicie reguły wprowadzania i reguły eliminacji. Reguły wprowadzania (czytane z góry na dół) mówią, jak udowodnić zdanie zawierające dany spójnik, np. $\lor\text{-intro}\ L$ mówi, że jeżeli z kontekstu $\Gamma$ wynika zdanie $A$, to z $\Gamma$ wynika też zdanie $A \lor B$. \\~\
	
	Reguły eliminacji (czytane z góry na dół) mówią, co możemy wywnioskować, mając dane zdanie w kontekście jako założenie, np. reguła $\land\text{-elim} R$ mówi, że jeżeli wiemy, że z $\Gamma$ wynika $A \land B$, to z $\Gamma$ wynika także $B$.
\end{frame}

\begin{frame}{Harmonia}
	Rozsądnym jest, żebyśmy eliminując dane zdanie za pomocą reguły eliminacji byli w stanie wywnioskować dokładnie tyle informacji, ile potrzebowaliśmy do jego udowodnienia za pomocą reguły wprowadzania. \\~\
	
	Dla przykładu, żeby udowodnić $A \land B$ musimy udowodnić osobno $A$ oraz $B$, więc mając $A \land B$ jako założenie możemy wywnioskować, że zachodzi nic więcej ponad $A$ i $B$. \\~\
	
	Zjawisko to nosi nazwę harmonii i jest charakterystyczne dla dedukcji naturalnej w logice intuicjonistycznej. W logice klasycznej nie jest ono obecne, gdyż tautologie takie jak $P \lor \neg P$ dają nam informacje za darmo.
\end{frame}

\begin{frame}{Dedukcja naturalna - reszta reguł}
	W systemie dedukcji naturalnej mamy też reguły podobne do tych znanych z rachunku sekwentów:
	\begin{itemize}
		\item Reguły strukturalne pozwalające nam usuwać, kopiować i przestawiać założenia obecne w kontekście.
		\item Regułę pozwalającą nam skorzystać z założenia.
		\item Regułę Cut pozwalającą udowodnić na boku przydatny lemat.
	\end{itemize}

\end{frame}

\begin{frame}{Dedukcja naturalna - podsumowanie}
	Dedukcja naturalna jest tym systemem dowodzenia, z którego większość ludzi czerpie swoją nieformalną semantykę dla logiki intuicjonistycznej. Nadaje się ona całkiem nieźle do automatycznego dowodzenia, choć nie aż tak dobrze jak rachunek sekwentów. \\~\
	
	Jej główną zaletą jest jednak podobieństwo do codziennego sposobu dowodzenia stosowanego przez matematyków oraz fakt, że jej dowody odpowiadają w bardzo ścisłym sensie termom rachunku $\lambda$ z typami prostymi (korespondencja Curry'ego-Howarda), co czyni związane z nią intuicje podstawą rozmaitych systemów typów, funkcyjnych języków programowania oraz asystentów dowodzenia. \\~\
	
	Więcej o dedukcji naturanej bez kontekstów można przeczytać tutaj: \url{http://www.cs.cmu.edu/~fp/courses/15317-f17/lectures/02-natded.pdf}
\end{frame}

\section{Korespondencja Curry'ego-Howarda}

\begin{frame}{Korespondencja Curry'ego-Howarda}

\begin{columns}
\begin{column}{0.5\textwidth}
\begin{itemize}
	\item Typ $1$ (singleton).
	\item Typ $\emptyset$ (pusty).
	\item Termami produktu $A \times B$ są pary termów $(a, b)$
	\item Termami typu $A \to B$ są funkcje, które przekształcają term typu $A$ w term typu $B$.
\end{itemize}
\end{column}
\begin{column}{0.5\textwidth}
\begin{itemize}
	\item Zdanie $\top$ (prawda).
	\item Zdanie $\bot$ (fałsz).
	\item Dowodami koniunkcji $P \land Q$ są pary dowodów $(p, q)$.
	\item Dowód implikacji $P \impl Q$ przekształca dowód poprzednika $P$ w dowód następnika $Q$. 
\end{itemize}
\end{column}
\end{columns}
\end{frame}

\begin{frame}{Korespondencja Curry'ego-Howarda}
	Na rozwinięcie tematu nie wystarczyło czasu. Więcej można przeczytać tutaj: \url{http://www.cs.cmu.edu/~fp/courses/15317-f17/lectures/03-pap.pdf}
\end{frame}

\begin{frame}{Coq}
	Coq to funkcyjny język programowania z typami zależnymi oraz asystent dowodzenia (ang. proof assistant), którego działanie opiera się na ideach pochodzących ostatecznie od logiki intuicjonistycznej, dedukcji naturalnej i korespondencji Curry'ego-Howarda. \\~\
	
	Jednym z większych sukcesów w jego zastosowaniu była formalizacja dowodu twierdzenia o czterech barwach, znanego z teorii grafów. \\~\
	
	Jeżeli chcesz dowiedzieć się więcej na jego temat, zajrzyj na jego stronę domową \url{https://coq.inria.fr/} lub do mojej książki \url{https://zeimer.github.io/}
\end{frame}

\section{Zadania}

\begin{frame}{Zadania}
\begin{itemize}
	\item Znajdź algebrę Heytinga, w której nie zachodzi $P \lor \neg P$.
	\item Znajdź model Kripkego, w którym nie zachodzi $P \lor \neg P$.
	\item Udowodnij zdanie $\neg (P \land \neg P)$ w systemie Hilberta, rachunku sekwentów i dedukcji naturalnej.
	\item Czy dowód twierdzenia Cantora-Bernsteina jest konstruktywny?
	\item Sprawdź, które tautologie logiki klasycznej są tautologiami logiki intuicjonistycznej, a które nie.
	\item (Opcjonalne) Zainstaluj Coqa, przeczytaj \url{https://zeimer.github.io/R1.html} i wykonaj wszystkie ćwiczenia.
\end{itemize}
\end{frame}

\end{document}