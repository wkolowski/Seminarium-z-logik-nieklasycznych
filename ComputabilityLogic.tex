\documentclass{beamer}
\usepackage[utf8]{inputenc}
\usepackage{polski}
\usepackage[polish]{babel}
\usetheme{Darmstadt}

\newcommand{\impl}{\rightarrow}
\renewcommand{\implies}{\rightarrow}

\title{Computability Logic}
\author{Wojciech Kołowski}
\date{12 czerwca 2018}

\begin{document}

\frame{\titlepage}

\frame{\tableofcontents}

\begin{frame}{Ostrzeżenie}
\begin{itemize}
	\item Niestety zabrakło mi czasu, żeby dokończyć prezentację. Odsyłam w związku z tym do poniższej publikacji:
	\item \href{https://arxiv.org/pdf/1612.04513.pdf}{A Survey of Computability Logic} (gdyby komuś nie działał link, to dostępna jest pod adresem \url{https://arxiv.org/pdf/1612.04513.pdf}
\end{itemize}
\end{frame}

\section{Motywacja}

\begin{frame}{Słabe punkty logik}
\begin{itemize}
	\item Logika klasyczna: każda maszyna Turinga kończy pracę lub nie... ale czy da się napisać program (albo zbudować maszynę), która będzie to rozstrzygać? Potrzeba logiki intuicjonistycznej.
	\item Logika intuicjonistyczna: jeżeli mam 10 zł, to mogę kupić kebaba. Wobec tego za 10 zł mogę kupić dowolną ilość kebabów. Potrzeba logiki liniowej.
	\item Logika liniowa: być może w Smoleńsku był zamach. Potrzeba logiki modalnej.
	\item Logika modalna: zawsze lubię placki, więc jutro lubię placki. Potrzeba logiki temporalnej.
	\item Logika temporalna nie jest bynajmniej logicznym zbawicielem.
\end{itemize}
\end{frame}

\begin{frame}{Idee stojące za logikami}
	Jak widać, za każdą logiką stoi jakaś idea, do której ta logika się ogranicza, odrzucając (przynajmniej częściowo) inne.
	\begin{itemize}
		\item Logika klasyczna — prawda.
		\item Logika intuicjonistyczna — obliczenia.
		\item Logika liniowa — zasoby.
		\item Logiki modalne — modalności.
		\item Logiki temporalne — czasy.
	\end{itemize}
\end{frame}

\begin{frame}{Być jak Sauron}
\begin{itemize}
	\item Rodzi się naturalne pytanie: czy da się stworzyć Jedyny Pierścień, który będzie rządził innymi?
	\item W sumie moglibyśmy zsumować wszystkie wymienione wyżej logiki i gitara, c'nie?
	\item Nie do końca, bo musimy jeszcze ustalić, jakie są zależności między różnymi spójnikami, np. liniową implikacją i modalnościami.
	\item Jeżeli pozbędziemy się idei modalności i czasu, to pewien Gruzin nazwiskiem Japaridze twierdzi, że udało mu się zbudować Jedyny Pierścień.
\end{itemize}
\end{frame}

\section{Droga do celu}

\begin{frame}{Semantyka ponad składnią}
\begin{itemize}
	\item Skoro naiwne podejście składniowe (``wrzućmy do jednego wora spójniki z różnych logik i zobaczmy, co wyjdzie'') nie zadziała, to pozostaje nam podejście semantyczne. \\
	\item Należy zaznaczyć, że Japaridze uważając semantykę za ważniejszą od składni popełnia błąd — semantyka i składnia są równie istotne.
\end{itemize}
\end{frame}

\begin{frame}{Retoryka i prawda}
\begin{itemize}
	\item Logikę można postrzegać jako pewną grę: oto dwóch retorów przerzuca się argumentami, próbując dowieść swoich racji.
	\item Jeżeli przyjmiemy, że podzielają oni pewne standardy dotyczące tego, jakie argumenty są przekonujące, to możemy powiedzieć, że zdanie jest prawdziwe, gdy w odpowiadającej mu grze jeden z retorów zawsze jest w stanie przekonać drugiego do swojego zdania niezależnie od tego, jakich ten użyje kontrargumentów.
\end{itemize}
\end{frame}

\begin{frame}{Obliczenia interaktywne}
\begin{itemize}
	\item Obliczenia klasycznie rozumiane dotyczą funkcji, tj. maszyna dostaje coś na wejściu i ma zwrócić na wyjściu odpowiedź.
	\item Nie odpowiada to wielu sytuacjom spotykanym w codziennym życiu, np. dialog użytkownika z serwerem.
	\item Można wprawdzie modelować takie sytuacje za pomocą funkcji, ale daje to marne efekty. Np. serwer nie staje się wolniejszy wraz z każdym zapytaniem, co znaczy, że jego odpowiedzi nie są wynikami funkcji zależącej od historii interakcji z użytkownikiem. 
	\item Jest tak dlatego, że obliczanie funkcji charakteryzuje się niskim poziomem interaktywności — interakcja sprowadza się tu do jednego zapytania i jednej odpowiedzi.
\end{itemize}
\end{frame}

\begin{frame}{Obliczenia interaktywne 2}
\begin{itemize}
	\item Obliczenia interaktywne również można postrzegać jako pewien rodzaj gry.
	\item Jeden z graczy podaje na wejściu swoje zapytanie, a drugi oblicza odpowiedź. Każdy z nich może korzystać z poprzednich zapytań i odpowiedzi.
	\item Możemy powiedzieć, że problem jest obliczalny, jeżeli w odpowiadającej mu grze gracz liczący ma strategię, która zawsze zwraca poprawne rozwiązanie problemu, niezależnie od tego, jakie dane drugi gracz poda na wejściu.
	\item Zauważmy, że zdania logiki klasycznej są specjalnym przypadkiem problemów obliczeniowych o zerowym stopniu interakcji — są prawdą lub fałszem i nie trzeba tu nic liczyć.
\end{itemize}
\end{frame}

\begin{frame}{Game semantics}
\begin{itemize}
	\item Odpowiednią semantyką dla naszej logiki są zatem gry.
	\item Jednym z graczy w naszych grach będzie maszyna, reprezentująca wykonywanie obliczeń. Drugim będzie środowisko, którego celem będzie uprzykrzanie maszynie liczenia przez zadawanie głupich pytań i podawanie dziwnych danych.
	\item Aby całość nie była oszukana, nakładamy na maszynę obowiązek grania jedynie według strategii algorytmicznych. Środowisko może stosować dowolne strategie.
\end{itemize}
\end{frame}

\begin{frame}{Game semantics 2}
\begin{itemize}
	\item Teoria gier, jako studium kooperacji i konfliktów, bardzo przydatne w ekonomii, może wydawać się dobrym narzędziem do myślenia o świecie rzeczywistym.
	\item Faktycznie, gry o charakterze formalnym dobrze opisują różne pojedyncze zjawiska zachodzące w świecie.
	\item Japaridze myli się jednak twierdząc, że logika jest najpełniejszym, spójnym, naturalnym, adekwatnym i wygodnym narzędziem do kierowania swoimi poczynaniami w życiu.
\end{itemize}
\end{frame}

\begin{frame}{Dualność: problemy i zasoby}
\begin{itemize}
	\item A co z zasobami?
	\item Nasze gry możemy interpretować jako problemy obliczeniowe, które maszyna musi rozwiązać.
	\item Jeżeli maszyna rozwiązuje problem, to środowisko może się jej przyglądać i wyciągać wnioski z tego procederu.
	\item Podobnie gdybyśmy zamienili graczy stronami i to środowisko musiało rozwiązać problem, maszyna mogłaby skorzystać z tego rozwiązania.
	\item Prowadzi nas to do sformułowania pewnej dualności: problemy jednego z graczy to zasoby drugiego. Schadenfreude!
\end{itemize}
\end{frame}

\begin{frame}{CoL i inne logiki}
\begin{itemize}
	\item Przez skupienie się na ideach prawdy, obliczeń i zasobów CoL jest powiązana z logikami klasyczną, intuicjonistyczną i liniową.
	\item CoL jest konserwatywnym rozszerzeniem logiki klasycznej.
	\item Pewien fragment CoL jest podobny do logiki liniowej, ale występują różnice. Ogólnie zdania logiki afinicznej są słuszne w CoL, ale nie na odwrót.
	\item Pewien fragment CoL jest bardzo podobny do logiki intuicjonistycnzej, ale występują różnice.
	\item Wynika to z różnego podejścia tych logik: semantycznego w przypadku CoL i syntaktycznego w przypadku logiki liniowej oraz intuicjonistycznej.
\end{itemize}
\end{frame}

\begin{frame}{Ekumenizm logiczny}
\begin{itemize}
	\item Logika klasyczna wszystkie wystąpienia zdania $P$ traktuje jako ten sam zasób, zaś logika liniowa każde wystąpienie $P$ traktuje jako osobny zasób. CoL pozwala wyrazić całe spektrum możliwości.
	\item O ile logika klasyczna i intuicjonistyczna pozostają w konflikcie, o tyle CoL ma charakter ekumeniczny: mamy i jedno i drugie...
	\item ... i jeszcze więcej: w ramach bonusu CoL jest również konserwatywnym rozszerzeniem logiki IF (\textit{independence friendly logic}).
\end{itemize}
\end{frame}


\section{Gry}

\begin{frame}{Pojęcia}
\begin{itemize}
	\item Gra to drzewko, które nazywamy \textit{gamestructure} (w wesołym tłumaczeniu: wydmuszkowa gra) i oznaczamy $\textbf{Lr}$.
	\item Wierzchołek drzewa to \textit{run} (oznaczany $\Gamma, \Delta$).
	\item Wierzchołek w skończonej odległości od korzenia to pozycja (oznaczana $\Phi, \Psi, \Xi, \Omega$).
	\item Wierzchołki są utożsamiane z prowadzącymi do nich ciągami ruchów. $\langle\rangle$ to pozycja pusta.
	\item Nazwa krawędzi ($\alpha, \beta, \gamma$) to ruch.
	\item Nazwa krawędzi wraz z informacją o graczu, który wykonuje ruch ($\textcolor{green}{\alpha}, \textcolor{green}{\beta}, \textcolor{green}{\gamma}$ dla maszyny i $\textcolor{red}{\alpha}, \textcolor{red}{\beta}, \textcolor{red}{\gamma}$ dla środowiska) to kolorowy ruch. W czarno-białym świecie oznaczane $\top\alpha, \top\beta, \top\gamma$ dla maszyny i $\bot\alpha, \bot\beta, \bot\gamma$ dla środowiska.
\end{itemize}
\end{frame}

\begin{frame}{Pojęcia 2}
\begin{itemize}
	\item \textit{Content} to funkcja $\textbf{Wn}: \textbf{Lr} \to \{\top, \bot\}$. Mówi on, który gracz wygrywa grę w danej pozycji: jeśli $\textbf{Wn}(\Gamma) = \top$, to wygrywa maszyna. W przeciwnym wypadku wygrywa środowisko.
	\item Gra stała (\textit{constant game}) to para $(\textbf{Lr}, \textbf{Wn})$, gdzie pierwszy komponent to \textit{gamestructure}, a drugi — \textit{content}. Gry stałe uogólniają zdania znane z logiki klasycznej.
	\item Gra stała jest \textit{strict}, gdy dla każdej pozycji co najwyżej jeden gracz możę wykonać ruch. Gra, która nie jest \textit{strict}, jest \textit{free}.
	\item Co ciekawe, większość ciekawych gier jest \textit{free}. Motywacja: jeżeli gramy w dwie gry na raz, to ruchy w obu grach mogą być w dowolnej kolejności.
	\item Strategie dla gier \textit{free} nie mogą być funkcjami z pozycji w ruchy.
\end{itemize}
\end{frame}

\begin{frame}{Pojęcia 3}
\begin{itemize}
	\item Gra jest skończonej długości (\textit{finite depth}) gdy długość gałęzi jest ograniczona przez liczbę $d \in \mathbb{N}$.
	\item Gra jest \textit{perifinite-depth} gdy długość gałęzi jest skończona (ale może być nieograniczona).
	\item Gra jest skończonej szerokości (\textit{finite breadth}), gdy ma skończoną ilość liści (pozycji końcowych).
	\item Gra jest skończona (\textit{finite}), gdy ma skończoną liczbę pozycji.
	\item Gra elementarna (\textit{elementary game}) to gra o głębokości 0.
	\item Istnieją dokładnie dwie elemenarne gry stałe: $\top$ (prawda, problem trywialny) i $\bot$ (fałsz, problem nierozwiązywalny).
\end{itemize}
\end{frame}

\begin{frame}{Przykład}
\begin{itemize}
	\item Z niedostatku technologicznego przykład jest niewidzialny (gra odpowiadająca obliczaniu następnika).
\end{itemize}
\end{frame}

\begin{frame}{Dlaczego obliczenia interaktywne}
\begin{itemize}
	\item Wiele problemów ma więcej niż jedno rozwiązanie, np. problem ``oblicz liczbę większą od $n$''.
	\item Wiele problemów może na tym samym poziomie mieć różny kolor pozycji. Np. dla niektórych wejść funkcja częściowa jest niezdefiniowana.
	\item Więcej niż 2 poziomy to gry bardziej interaktywne.
\end{itemize}
\end{frame}

\begin{frame}{Pojęcia 4}
\begin{itemize}
	\item Gry nie muszą być stałe — mogą zależeć od zmiennych. Zapis: $G(x)$.
	\item Jeżeli $G(x)$ jest grą, $x$ zmienną, a $c$ stałą, to $G(c)$ jest instancją gry $G(x)$.
	\item Gra $G(x)$ jest elementarna, jeżeli wszystkie jej instancje są elementarne.
	\item Gry stałe to np. szachy (bez remisów), warcaby etc.
	\item Gry niestałe to gry karciane — zależą one od początkowej permutacji kart.
	\item Pozycja jest unilegalna (\textit{unilegal}) gdy jest legalna we wszystkich instancjach gry.
	\item Gra $G(x)$ jest unistrukturalna (\textit{unistructural}) gdy wszystkie pozycje wszystkich jej instancji są unilegalne. Intuicja: dla każdej stałej $c$ drzewko $G(c)$ ma taki sam kształt.
\end{itemize}
\end{frame}

\begin{frame}{Pojęcia 5}
\begin{itemize}
	\item Gra stała jest statyczna, gdy zwycięstwo zależy od strategii, a nie od szybkości. Jeżeli gra nie jest statyczna, to jest dynamiczna.
	\item Gra jest statyczna, jeżeli wszystkie jej instancje są statyczne.
	\item Wszystkie operacje na grach, z którymi się zapoznamy, zachowują statyczność.
\end{itemize}
\end{frame}

\begin{frame}{Teza (Church-Turing-Japaridze)}
\begin{itemize}
	\item Gry statyczne dobrze formalizują intuicyjne pojęcie interaktywnego (niezależnego od szybkości) problemu obliczeniowego.
	\item Istnienie strategii wygrywającej dobrze formalizuje intuicyjne pojęcie algorytmicznego rozwiązania interaktywnego problemu obliczeniowego.
\end{itemize}
\end{frame}

\end{document}